\documentclass{article}
\usepackage[hyphens]{url}
\usepackage{hyperref}
\usepackage{url}
\usepackage{amsmath}
\usepackage{amsfonts}
\usepackage{amsthm}
\usepackage{enumerate}
\usepackage{listings}
\usepackage{graphicx}
\usepackage{geometry}
%\usepackage{algorithm}
%\usepackage{algorithmic}
\geometry{a4paper,left=10mm,right=10mm,top=20mm,bottom=20mm}

\title{Seminar: Hints}
\date{}
\author{Stefanie Roos}

\begin{document}
\maketitle

\textbf{Searching for further papers: }
\begin{itemize}
\item search for key words on Google Scholar (\url{scholar.google.com})
\item look at the references for papers on the topic
\item look at papers that cited a relevant paper (\url{http://dl.acm.org/})
\item prefer papers from good conferences: INFOCOM, SIGCOMM, CCS, S\&P (Oakland), WWW, Usenix, PETs, STOC, FOCS,... (\url{http://en.wikipedia.org/wiki/List_of_computer_science_conferences})
\end{itemize}

\textbf{Reading papers: }
\begin{itemize}
\item decide on a set of questions you want to be answered by this paper
\item write down/mark the answers to those questions + other interesting stuff you find
\item relate the paper to others, categorization 
\item you can (but don't have to) use tools for organizing literature (e.g. Mendeley \url{http://www.mendeley.com/}) 
\end{itemize}

\textbf{Writing the final version: }
\begin{itemize}
\item decide on structure: 
   \begin{itemize}
    \item abstract, introduction and conclusion are necessary
    \item possibly a background section introducing concepts necessary to understand the actual content: notation, mathematical concepts and theorems, basic terms
    \item the remaining structure should classify the papers you read in some way
    \item most cases: one general section introducing the different categories and one (sub)section per category
    \item alternative: discuss analytical ideas in one section, practical evaluation in a second section
    \item alternative: discuss ideas in one section and results in a second
   \end{itemize} 
\item write each section down in bullet points usually starting with the 'content sections', organise bullet points into groups, so that each group contains one basic thought (which will later be one paragraph)
\item write down the full text
\item several iterations of corrections:
\begin{enumerate}
\item structure, overall content (ideally before writing down the full text!)
\item content, sense, understanding
\item grammar and spelling
\end{enumerate}
\end{itemize}

\textbf{LaTeX:}
\begin{itemize}
\item compile regularly, at latest after each paragraph, after each (long) mathematical equation and whenever you try something new
\item behavior varies between different operating systems, LaTeX distributions and Editors, so make sure you know your system setting when googling for help
\item Guide: \url{http://en.wikibooks.org/wiki/LaTeX}
\item Forum: \url{http://www.latex-community.org/forum/}
\end{itemize}



\end{document}